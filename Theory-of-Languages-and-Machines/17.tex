\documentclass[10pt,a4paper]{article}

\usepackage{graphicx}
% Requires \usepackage{graphicx}

\begin{document}

\scriptsize

\begin{flushright}
\textsf{Basic Terminology}  \textbf{$|$ 17}
\end{flushright}


\begin{flushleft}
8. A binary tree is a tree in which each node has at most two children.

9. A full binary tree is a tree in which every node other than the leaves has two children.

10. The operations of synchronous circuit are controlled by clock pulses. The operations in an asynchronous circuit are controlled by a number of completion and initialization signals. Here, the
completion of one operation is the initialization of the execution of the next consecutive operation.

11. The circuit whose output depends on the present state only is called a combinational circuit.,i.e.,O/P = Func.(Present I/P).

12. If the output is the function of the external input and the present stored information, then the circuit
is called a sequential circuit.,i.e.,O/P=Func.(External I/P and Present stored information).
\end{flushleft}

\begin{figure}[h]
  \centering
  % Requires \usepackage{graphicx}
  \includegraphics[width=12cm]{1.1}\\
\end{figure}

\begin{flushleft}
  1. Prove that A\, $\cup \,( B \,\cap \,C ) = ( A \,\cup \,B ) \,\cap\, ( A \,\cup \,C )$.
\end{flushleft}

\textbf{\textsl{Solution:}} Let $P = A \,\cup\, (B \,\cup\, C)  \,and \, Q = (A \,\cup\, B) \,\cap\, (A \,\cup\, C).$

\qquad Consider an element a\, $\in$ \,P and b\, $\in$ \,Q.

\qquad As $P = A \,\cup\, (B \,\cap\, C) and \, \:a \in P,$ it implies that
\begin{center}
$\quad \qquad a \in A\: or \:a \in (B \,\cap\, C)$

$\quad \,\qquad\Longrightarrow a \in A \:or \{a \in B \:and \:a \in C\}$

$\qquad \qquad\qquad\qquad\Longrightarrow \{a \in A \:or\: a \in B\} and \{a \in A \:or\: a \in C\}$

$\quad\quad\quad\:\,\Longrightarrow a \in (A \cup B) and \:a \in (A \cup C)$

$\quad\Longrightarrow a \in (A \cup B) \cap (A \cup C)$
\end{center}

It is proved that P is a subset of Q.

As $Q = (A \cup B) \cap (A \cup C) and b \in Q,$ it implies that
\begin{center}
$\quad \qquad  b \in (A \cup B) and\:\: b \in (A \cup C) $

$\quad\qquad \,\qquad\quad\qquad\Longrightarrow \{b \in A\: or\: b \in B\} and \{b \in A \: or\: b \in C\}$

$\quad\quad\:\quad\:\,\Longrightarrow b \in A\: or \{b \in B \:and \:\:b \in C\}$

$\quad\Longrightarrow b \in A \:or\: b \in (B \cap C)$
\end{center}

It is proved that Q is a subset of P.

As P is a subset of Q and Q is a subset of P, it implies that P $\equiv$ Q.

Thus $A \cup (B \cap C) = (A \cup B) \cap (A \cup C)$ is proved.

\begin{flushleft}
    \qquad 2. Prove the following identity:

  \qquad r (s + t) = rs + rt
\end{flushleft}

\textsl{\textbf{Solution:}} Let $x \in r(s + t),$

\begin{center}

  $\qquad\qquad\Longrightarrow x \in r \;and\; x \in (s + t)$
  
  $\qquad\qquad\quad\quad\Longrightarrow x \in r \;and\: (x \in s \:or\: x \in t)$
  
  $\qquad\qquad\qquad\qquad\qquad\quad\Longrightarrow (x \in r \;and\; x \in s) or (x \in r \;and\; x \in t)$
  
  $\:\;\Longrightarrow x \in (rs + rt)$
  
\end{center}

That is $r(s + t) \subseteq (is a subset of) (rs + rt)$.




\end{document}
