\documentclass [7pt]{beamer}
\usepackage{xcolor}	
\usepackage{tikz}
\usetheme{Frankfurt}
\useoutertheme{infolines}
\usepackage{ragged2e}
\usepackage{amsmath}
\usepackage{amssymb}
\begin{document}
\small
\section*{mesal 1 , 2}
\subsection*{Sina akbari }	
\begin{frame}
\justifying	
1. Prove that $A \cup(B \cap C)=(A \cup B) \cap(A \cup C)$.
\begin{flushleft}
Solution: Let $P=A \cup(B \cup C)$ and $Q=(A \cup B) \cap(A \cup C)$.

Consider an element a $\in \mathrm{P}$ and $\mathrm{b} \in \mathrm{Q}$. 


As $P=A \cup(B \cap C)$ and $a \in P,$ it implies that
\end{flushleft}
$$
\begin{array}{l}
	\quad a \in A \text { or } a \in(B \cap C) \\
	\Longrightarrow a \in A \text { or }\{a \in B \text { and } a \in C\} \\
	\Longrightarrow\{a \in A \text { or } a \in B\} \text { and }\{a \in A \text { or } a \in C\} \\
	\Longrightarrow a \in(A \cup B) \text { and } a \in(A \cup C) \\
	\Longrightarrow a \in(A \cup B) \cap(A \cup C)
\end{array}
$$
It is proved that $\mathrm{P}$ is a subset of $\mathrm{Q}$. 


As $Q=(A \cup B) \cap(A \cup C)$ andb $\in Q,$ it implies that
$$
\begin{array}{l}
	b \in(A \cup B) a n d b \in(A \cup C) \\
	\Longrightarrow\{b \in A \text { or } b \in B\} a n d\{b \in A \text { or } b \in C\} \\
	\Longrightarrow b \in A \text { or }\{b \in B \text { and } b \in C\} \\
	\Longrightarrow b \in A \text { or } b \in(B \cap C)
\end{array}
$$
It is proved that $Q$ is a subset of $P$.


As $P$ is a subset of $Q$ and $Q$ is a subset of $P,$ it implies that $P \equiv Q$. 


Thus $A \cup(B \cap C)=(A \cup B) \cap(A \cup C)$ is proved.
\end{frame}


\begin{frame}	
2. Prove the following identity:


r(s+t)=r s+r t
\begin{flushleft}
Solution: Let $x \in r(s+t)$,
\end{flushleft}
$$
\begin{array}{l}
\Longrightarrow x \in r \text { and } x \in(s+t) \\
\Longrightarrow x \in r \text { and }(x \in s \text { or } x \in t) \\
\Longrightarrow(x \in r \text { and } x \in s) o r(x \in r \text { and } x \in t) \\
\Longrightarrow x \in(r s+r t)
\end{array}
$$
That is $r(s+t) \subseteq($isasubsetof$)(r s+r t)$.
\end{frame}
\begin{frame}
\justifying	
Let $x \in(r s+r t)$
$$
\begin{array}{l}
\Longrightarrow(x \in r s \text { or } x \in r t) \\
\Longrightarrow(x \in r \text { and } x \in s) \text { or }(x \in r \text { and } x \in t) \\
\Longrightarrow x \in r \text { and }(x \in \text { s or } x \in t) \\
\Longrightarrow x \in r(s+t)
\end{array}
$$
That is $(r s+r t) \subseteq r(s+t)$.


So, $r(s+t)=r s+r t($ proved $)$ .
\end{frame}
\end{document}